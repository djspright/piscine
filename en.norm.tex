\documentclass{42-en}



%******************************************************************************%
%                                                                              %
%                               Prologue                                       %
%                                                                              %
%******************************************************************************%

\begin{document}
\title{The Norm}
\subtitle{Version 4.1}

\summary
{
    This document describes the applicable standard (Norm) at 42: a
    programming standard that defines a set of rules to follow when writing code.
    The Norm applies to all C projects within the Common Core by default, and
    to any project where it's specified.
}

\maketitle

\tableofcontents



%******************************************************************************%
%                                                                              %
%                                 Foreword                                     %
%                                                                              %
%******************************************************************************%
\chapter{Foreword}

The \texttt{norminette} is a Python and open source code that checks Norm
compliance of your source code. It checks many constraints of the Norm, but not
all of them (eg. subjective constraints). Unless specific local regulations on
your campus, the \texttt{norminette} prevails during evaluations on the controlled
items. In the following pages, rules that are not checked by the \texttt{norminette}
are marked with \textit{(*)}, and can lead to project failure (using the Norm flag) if
discovered by the evaluator during a code review.\\

Its repository is available at https://github.com/42School/norminette.\\

Pull requests, suggestions and issues are welcome!

\newpage


%******************************************************************************%
%
%                                   Pedago explanations                        %
%
%******************************************************************************%
    \chapter{Why?}

    The Norm has been carefully crafted to fulfill many pedagogical needs. Here
    are the most important reasons for all the choices below:
    \begin{itemize}

    \item Sequencing: coding implies splitting a big and complex task into a long series
      of elementary instructions. All these instructions will be executed in sequence:
      one after another. A beginner that starts creating software needs a simple and clear
      architecture for their project, with a full understanding of all individual instructions
      and the precise order of execution. Cryptic language syntaxes that do multiple
      instructions apparently at the same time are confusing, functions that try to address
      multiple tasks mixed in the same portion of code are source of errors.\\
      The Norm asks you to create simple pieces of code, where the unique task of each piece
      can be clearly understood and verified, and where the sequence of all the executed
      instructions leaves no doubt. That's why we ask for 25 lines maximum in functions, also why
      \texttt{for}, \texttt{do .. while}, or ternaries are forbidden.

    \item Look and Feel: while exchanging with your friends and workmates during the
      normal peer-learning process, and also during the peer-evaluations, you do not
      want to spend time to decrypt their code, but directly talk about the
      logic of the piece of code.\\
      The Norm asks you to use a specific look and feel, providing instructions for the naming
      of the functions and variables, indentation, brace rules, tab and spaces at many places... .
      This will allow you to smoothly have a look at other's codes that will look familiar,
      and get directly to the point instead of spending time reading the code before understanding it.
      The Norm also comes as a trademark. As part of the 42 community, you will be able to
      recognize code written by another 42 student or alumni when you'll be in the labor market.

    \item Long-term vision: making the effort to write understandable code is the
      best way to maintain it. Each time that someone else, including you, has to fix a bug
      or add a new feature they won't have to lose their precious time trying to figure out
      what it does if previously you did things in the right way. This will avoid situations
      where pieces of code stop being maintained just because it is time-consuming, and that
      can make the difference when we talk about having a successful product in the market.
      The sooner you learn to do so, the better.

    \item References: you may think that some, or all, the rules included on the Norm are
      arbitrary, but we actually thought and read about what to do and how to do it. We highly
      encourage you to Google why the functions should be short and just do one thing, why the
      name of the variables should make sense, why lines shouldn't be longer than 80 columns wide,
      why a function should not take many parameters, why comments should be useful, etc.

    \end{itemize}


\newpage

%******************************************************************************%
%                                                                              %
%                                The Norm                                      %
%                                                                              %
%******************************************************************************%
\chapter{The Norm}


%******************************************************************************%
%                             Naming conventions                               %
%******************************************************************************%
    \section{Naming}

        \begin{itemize}

            \item A structure's name must start by
                \texttt{s\_}.

            \item A typedef's name must start by
                \texttt{t\_}.

            \item A union's name must start by \texttt{u\_}.

            \item An enum's name must start by \texttt{e\_}.

            \item A global's name must start by \texttt{g\_}.

            \item Identifiers, like variables, functions names, user defined types,
              can only contain lowercases, digits and '\_' (snake\_case). No capital letters are allowed.

            \item Files and directories names can only contain lowercases, digits and
                '\_' (snake\_case).

            \item Characters that aren't part of the standard
                ASCII table are forbidden, except inside litteral strings and chars.

            \item \textit{(*)} All identifiers (functions, types,
              variables, etc.) names should be explicit, or a mnemonic,
              should be readable in English, with each word separated by an underscore.
              This applies to macros, filenames and directories as well.

            \item Using global variables that are not marked const or static is
            forbidden and is considered a norm error, unless the project explicitly allows them.

            \item The file must compile. A file that doesn't compile isn't expected
                to pass the Norm.
        \end{itemize}
\newpage

%******************************************************************************%
%                                 Formatting                                   %
%******************************************************************************%
    \section{Formatting}

            \begin{itemize}

            \item Each function must be at most 25 lines long, not
              counting the function's own braces.

            \item Each line must be at most 80 columns wide, comments
              included. Warning: a tabulation doesn't count
              as a single column, but as the number of spaces it
              represents.

            \item Functions must be separated by an empty line. Comments or preprocessor instructions
              can be inserted between functions. At least an empty line must exists.

            \item You must indent your code with 4-char-long tabulations.
              This is not the same as 4 spaces, we're talking about real tabulations here (ASCII char number 9).
              Check that your code editor is correctly configured in order to visually get a proper indentation
              that will be validated by the \texttt{norminette}.

            \item Blocks within braces must be indented. Braces are alone on their own line,
              except in declaration of struct, enum, union.

            \item An empty line must be empty: no spaces or tabulations.

            \item A line can never end with spaces or tabulations.
              
            \item You can never have two consecutive empty lines.
              You can never have two consecutive spaces.

            \item Declarations must be at the beginning of a function.

            \item All variable names must be indented on the same
              column in their scope. Note: types are already indented by the containing block.

            \item The asterisks that go with pointers must be stuck to
              variable names.

            \item One single variable declaration per line.

            \item Declaration and an initialisation cannot be
              on the same line, except for global variables (when allowed),
              static variables, and constants.

            \item In a function, you must place an empty line between
              variable declarations and the remaining of the function.
              No other empty lines are allowed in a function.

            \item Only one instruction or control structure per line is allowed. Eg.: Assignment in
              a control structure is forbidden, two or multiple assignments on the same line is forbidden,
              a newline is needed at the end of a control structure, ... .

            \item An instruction or control structure can be split into multiple lines when needed.
              The following lines created must be indented compared to the first line,
              natural spaces will be used to cut the line, and if applies, operators will be
              at the beginning of the new line and not at the end of the previous one.

            \item Unless it's the end of a line, each comma or semi-colon
              must be followed by a space.

            \item Each operator or operand must be separated by one
              - and only one - space.

            \item Each C keyword must be followed by a space, except for
              keywords for types (such as int, char, float, etc.),
              as well as sizeof.

            \item Control structures (if, while..) must use braces, unless they contain a single
              instruction on a single line.

            \end{itemize}

\vspace{1cm}

            General example:
            \begin{42ccode}
int             g_global;
typedef struct  s_struct
{
    char    *my_string;
    int     i;
}               t_struct;
struct          s_other_struct;

int     main(void)
{
    int     i;
    char    c;

    return (i);
}
            \end{42ccode}
            \newpage

%******************************************************************************%
%                              Function parameters                             %
%******************************************************************************%
    \section{Functions}

        \begin{itemize}

            \item A function can take 4 named parameters at most.

            \item A function that doesn't take arguments must be
                explicitly prototyped with the word "void" as the
                argument.

            \item Parameters in functions' prototypes must be named.

            \item You can't declare more than 5 variables per function.

            \item Return of a function has to be between parenthesis, unless the
              function returns nothing.

            \item Each function must have a single tabulation between its
                return type and its name.

        \end{itemize}

\vspace{1cm}
        
            \begin{42ccode}
int my_func(int arg1, char arg2, char *arg3)
{
    return (my_val);
}

int func2(void)
{
    return ;
}
            \end{42ccode}

        \newpage


%******************************************************************************%
%                        Typedef, struct, enum and union                       %
%******************************************************************************%
    \section{Typedef, struct, enum and union}

        \begin{itemize}

        \item As other C keywords, add a space between ``struct'' and the name
          when declaring a struct. Same applies to enum and union.

        \item When declaring a variable of type struct, apply the usual indentation for the name
          of the variable. Same applies to enum and union.

        \item Inside the braces of the struct, enum, union, regular indentation rules
          apply, like any other blocks.

        \item As other C keywords, add a space after ``typedef'',
          and apply regular indentation for the new defined name.

        \item You must indent all structures' names on the same column for their scope.

        \item You cannot declare a structure in a .c file.

        \end{itemize}
        \newpage


%******************************************************************************%
%                                   Headers                                    %
%******************************************************************************%
    \section{Headers - a.k.a include files}

        \begin{itemize}

            \item \textit{(*)} The allowed elements of a header file are:
                header inclusions (system or not), declarations, defines,
                prototypes and macros.

            \item All includes must be at the beginning of the file.

            \item You cannot include a C file in a header file or another C file.

            \item Header files must be protected from double inclusions. If the file is
            \texttt{ft\_foo.h}, its bystander macro is \texttt{FT\_FOO\_H}.

            \item \textit{(*)} Inclusion of unused headers is forbidden.

            \item Header inclusion can be justified in the .c file and in the .h file itself
              using comments.

        \end{itemize}

\vspace{1cm}
        
        \begin{42ccode}
#ifndef FT_HEADER_H
# define FT_HEADER_H
# include <stdlib.h>
# include <stdio.h>
# define FOO "bar"

int     g_variable;
struct  s_struct;

#endif
        \end{42ccode}
        \newpage


%******************************************************************************%
%                                 The 42 header                                %
%******************************************************************************%

   \section{The 42 header - a.k.a start a file with style}

        \begin{itemize}

        \item Every .c and .h file must immediately begin with the standard 42 header:
          a multi-line comment with a special format including useful informations. The
          standard header is naturally available on computers in clusters for various
          text editors (emacs: using \texttt{C-c C-h}, vim using \texttt{:Stdheader} or
          \texttt{F1}, etc...).

        \item \textit{(*)} The 42 header must contain several informations up-to-date, including the
          creator with login and student email (@student.campus), the date of creation,
          the login and date of the last update. Each time the file is saved on disk,
          the information should be automatically updated.

        \end{itemize}
        \info{
          The default standard header may not automatically be configured with your personnal
          information. You may need to change it to follow the previous rule.
          }
        
        \newpage


%******************************************************************************%
%                           Macros and Pre-processors                          %
%******************************************************************************%
    \section{Macros and Pre-processors}

        \begin{itemize}

            \item \textit{(*)} Preprocessor constants (or \#define) you create must be used
                only for literal and constant values.
            \item \textit{(*)} All \#define created to bypass the norm and/or obfuscate
                code are forbidden.
            \item \textit{(*)} You can use macros available in standard libraries, only
                if those ones are allowed in the scope of the given project.
            \item Multiline macros are forbidden.
            \item Macro names must be all uppercase.
            \item You must indent preprocessor directives inside \#if, \#ifdef
                or \#ifndef blocks.
            \item Preprocessor instructions are forbidden outside of global scope.

        \end{itemize}
        \newpage


%******************************************************************************%
%                              Forbidden stuff!                                %
%******************************************************************************%
    \section{Forbidden stuff!}

        \begin{itemize}

            \item You're not allowed to use:

                \begin{itemize}

                    \item for
                    \item do...while
                    \item switch
                    \item case
                    \item goto

                \end{itemize}

            \item Ternary operators such as `?'.

            \item VLAs - Variable Length Arrays.

            \item Implicit type in variable declarations

        \end{itemize}

\vspace{1cm}
        
        \begin{42ccode}
    int main(int argc, char **argv)
    {
        int     i;
        char    str[argc]; // This is a VLA

        i = argc > 5 ? 0 : 1 // Ternary
    }
        \end{42ccode}
        \newpage

%******************************************************************************%
%                                   Comments                                   %
%******************************************************************************%
    \section{Comments}

        \begin{itemize}

        \item Comments cannot be inside function bodies.
          Comments must be at the end of a line, or on their own line

        \item \textit{(*)} Your comments should be in English, and useful.

        \item  \textit{(*)} A comment cannot justify the creation of a carryall or bad function.

        \end{itemize}

        \warn{
            A carryall or bad function usually comes with names that are
            not explicit such as f1, f2... for the function and a, b, c,..
            for the variables names.
            A function whose only goal is to avoid the norm, without a unique
            logical purpose, is also considered as a bad function.
            Please remind that it is desirable to have clear and readable functions that achieve a
            clear and simple task each. Avoid any code obfuscation techniques,
            such as the one-liner, ... .
        }
        \newpage


%******************************************************************************%
%                                    Files                                     %
%******************************************************************************%
    \section{Files}

        \begin{itemize}

            \item You cannot include a .c file in a .c file.

            \item You cannot have more than 5 function-definitions in a .c file.

        \end{itemize}
        \newpage


%******************************************************************************%
%                                   Makefile                                   %
%******************************************************************************%
    \section{Makefile}

            Makefiles aren't checked by the \texttt{norminette}, and must be checked during evaluation by
            the student when asked by the evaluation guidelines. Unless specific instructions, the following rules
            apply to the Makefiles:
            \begin{itemize}

                \item The \textit{\$(NAME)}, \textit{clean}, \textit{fclean}, \textit{re} and \textit{all}
                  rules are mandatory. The \textit{all} rule must be the default one and executed when typing just \texttt{make}.

                \item If the makefile relinks when not necessary, the project will be considered
                  non-functional.

                \item In the case of a multibinary project, in addition to
                  the above rules, you must have a rule for each binary (eg: \$(NAME\_1), \$(NAME\_2), ...).
                  The ``all'' rule will compile all the binaries, using each binary rule.

                  \item In the case of a project that calls a function from a non-system library
                  (e.g.: \texttt{libft}) that exists along your source code, your makefile must compile
                  this library automatically.

                  \item All source files needed to compile your project must
                    be explicitly named in your Makefile. Eg: no ``*.c'', no ``*.o'' , etc ...

            \end{itemize}


\end{document}
%******************************************************************************%
